%%%%%%%%%%%%%%%%%
% This is CV  creating using a template by LianTze Lim (liantze@gmail.com) template, 
% based on the CV created by BusinessInsider at 
% http://www.businessinsider.my/a-sample-resume-for-marissa-mayer-2016-7/?r=US&IR=T, and heavily modified
% by Zeshan Basaran (zeshan.basaran@gmail.com).
%
%% It may be distributed and/or modified under the
%% conditions of the LaTeX Project Public License, either version 1.3
%% of this license or (at your option) any later version.
%% The latest version of this license is in
%%    http://www.latex-project.org/lppl.txt
%% and version 1.3 or later is part of all distributions of LaTeX
%% version 2003/12/01 or later.
%%%%%%%%%%%%%%%%

\documentclass[10pt,a4paper,ragged2e,withhyper]{altacv}
%% AltaCV uses the fontawesome5 package.
%% See http://texdoc.net/pkg/fontawesome5 for full list of symbols.

% Change the page layout if you need to
\geometry{left=1.25cm,right=1.25cm,top=1.5cm,bottom=1.5cm,columnsep=1.2cm}

% The paracol package lets you typeset columns of text in parallel
\usepackage{paracol}

% Change the font if you want to, depending on whether
% you're using pdflatex or xelatex/lualatex
% WHEN COMPILING WITH XELATEX PLEASE USE
% xelatex -shell-escape -output-driver="xdvipdfmx -z 0" mmayer.tex
\ifxetexorluatex
  % If using xelatex or lualatex:
  \setmainfont{Roboto}
\else
  % If using pdflatex:
  \usepackage[default]{roboto}
\fi

% Change the colours if you want to
\definecolor{BrightBlue}{HTML}{1e90ff}
\definecolor{LightGrey}{HTML}{666666}
% \colorlet{name}{black}
% \colorlet{tagline}{PastelRed}
\colorlet{heading}{black}
\colorlet{headingrule}{black}
% \colorlet{subheading}{PastelRed}
\colorlet{accent}{BrightBlue}
\colorlet{emphasis}{black}
\colorlet{body}{black}

\begin{document}
\name{Zeshan Basaran}
\tagline{Computer Science Student}
\personalinfo{%
  % Can add more with \printinfo{symbol}{detail}
  % \printinfo{symbol}{detail}[optional hyperlink prefix]
  % \printinfo{\faPaw}{Hey ho!}

  %% Or you can declare your own field with
  %% \NewInfoFiled{fieldname}{symbol}[optional hyperlink prefix] and use it:
  % \NewInfoField{gitlab}{\faGitlab}[https://gitlab.com/]
  % \gitlab{your_id}
  %%
  
  \email{zeshan.basaran@gmail.com}
  \phone{443-605-8174}
  \location{Baltimore, MD}
  \homepage{zeshanbasaran.github.io/}
% \twitter{@zeshanbasaran}
  \linkedin{zeshanbasaran}
  \github{zeshanbasaran}

  %% For services and platforms like Mastodon where there isn't a
  %% straightforward relation between the user ID/nickname and the hyperlink,
  %% you can use \printinfo directly e.g.
  % \printinfo{\faMastodon}{@username@instace}[https://instance.url/@username]
  %% But if you absolutely want to create new dedicated info fields for
  %% such platforms, then use \NewInfoField* with a star:
  % \NewInfoField*{mastodon}{\faMastodon}
  %% then you can use \mastodon, with TWO arguments where the 2nd argument is
  %% the full hyperlink.
  % \mastodon{@username@instance}{https://instance.url/@username}
}

\makecvheader

%% Depending on your tastes, you may want to make fonts of itemize environments slightly smaller
\AtBeginEnvironment{itemize}{\small}

%% Set the left/right column width ratio to 6:4.
\columnratio{0.6}

% Start a 2-column paracol. Both the left and right columns will automatically
% break across pages if things get too long.
\begin{paracol}{2}

%-------------------------------------------------------------------------------
%	SECTION TITLE
%-------------------------------------------------------------------------------
\cvsection{Projects}


%-------------------------------------------------------------------------------
%	CONTENT
%-------------------------------------------------------------------------------
\cvevent{Translator (English-Binary-Morse)}{}{11/2023 - 12/2023}{Towson University}
\link{https://github.com/zeshanbasaran/Translator_Morse-Binary-English}{Translator (English-Binary-Morse) on Github}
Developed a Java program that translates messages between \\ English, Morse code, and ASCII binary code for my COSC 237 final project.
\begin{itemize}
\item Utilized concepts of subclasses, abstract classes, and polymorphism to practice proper code organization.
\end{itemize}

\divider

\cvevent{Movie Sentiment Analysis}{}{01/2023}{Personal Project}
\link{https://github.com/zeshanbasaran/movieSentimentAnalysis}{Movie Sentiment Analysis on Github}
Created a Python program that uses natural language processing (NLP) and machine learning algorithms from scikit-learn to \\ analyze a user's sentiment regarding a movie.
\begin{itemize}
\item Dataset was a collection of IMDB reviews from Kaggle.
\item It recieved 1000 positive and 1000 negative movie reviews from the dataset for training and testing.
\end{itemize}

\divider

\cvevent{Data Analysis for Sensors}{}{12/2022}{Allen Institute (Seattle, WA)}
\link{https://github.com/zeshanbasaran/SensorDataAnalysis}{Sensor Data Analysis on Github}
Composed a program in Python to organize data collected by the institute.
\begin{itemize}
\item Program was created for a connection at the institute, with extensive comments so the researcher could modify the program as necessary.
\item Organized 4,441 data points collected from glutamate sensors.
\end{itemize}
%-------------------------------------------------------------------------------
%	SECTION TITLE
%-------------------------------------------------------------------------------
\cvsection{Experience}


%-------------------------------------------------------------------------------
%	CONTENT
%-------------------------------------------------------------------------------
\cvevent{Server}{Cypriana}{08/2023 - Present}{Roland Park, MD}
\begin{itemize}
\item Provided high-quality service to elite guests, such as US senators and political candidates. 
\item Consistently received 5 star reviews and averaged over 20\% on tips.
\item Welcomed and mentored new team members.
\end{itemize}

\divider

\cvevent{To-Go Specialist}{Olive Garden}{08/2018 - 06/2023}{Westminster, MD}
\begin{itemize}
\item Took phone calls and prepared to-go and large catering orders while accommodating special requests.
\item Delivered exceptional customer service, contributing to our 99\% on-time delivery and 99\% order accuracy for to-go orders.
\item Successfully onboarded and trained 15-20 new team members, ensuring their seamless integration into the team and adherence to company standards.
\end{itemize}

\switchcolumn

%-------------------------------------------------------------------------------
%	SECTION TITLE
%-------------------------------------------------------------------------------
\cvsection{Education}


%-------------------------------------------------------------------------------
%	CONTENT
%-------------------------------------------------------------------------------
\cvevent{B.S. in Computer Science}{\textbf{Towson University}}{2022 - Present}{Towson, MD}
\begin{itemize}
\item GPA: 3.8 / 4.0
\item Expected Graduation: Spring 2025
\item Dean's List
\end{itemize}

\divider

\cvevent{Coursework toward B.S. in \\Engineering}{\textbf{Virginia Polytechnic Institute}}{2020 - 2021}{Blacksburg, VA (Online)}
\begin{itemize}
\item GPA: 3.1 / 4.0
\item Prerequisites such as Calculus 1 \& 2, Physics 101, and Basic Engineering Design
\end{itemize}

\divider

\cvevent{High School Diploma}{\textbf{Francis Scott Key High School}}{2016 - 2020}{Union Bridge, MD}
\begin{itemize}
\item GPA: 4.0 / 4.0
\item Honors and AP Classes
\item National Honors Society President
\item Environmental Club Advisor
\item Mock Trial Witness and Lawyer
\item Girls Soccer and Track
\item Speaking roles in 5 theater productions
\end{itemize}
%-------------------------------------------------------------------------------
%	SECTION TITLE
%-------------------------------------------------------------------------------
\cvsection{Volunteering}


%-------------------------------------------------------------------------------
%	CONTENT
%-------------------------------------------------------------------------------
\cvevent{Coding Club Workshop}{\textbf{Carroll County Public Library - Westminster Branch}}{A Week in June 2022}{}
Assisted a week-long coding workshop for middle and high schoolers.
\begin{itemize}
\item 
Provided one-on-one guidance and supported the participants as they worked through coding exercises and projects, leading to a 100\% completion rate for all 30 coding exercises.
\item 
Acted as a mentor and role model for 30 \\ aspiring programmers by fostering a positive and inclusive learning environment.
\end{itemize}

%-------------------------------------------------------------------------------
%	SECTION TITLE
%-------------------------------------------------------------------------------
\cvsection{Skills}


%-------------------------------------------------------------------------------
%	CONTENT
%-------------------------------------------------------------------------------
% Define a custom command for a bold and underlined skill item
\newcommand{\skillitem}[1]{%
    \textbf{\underline{#1}}%
    \quad
}

% Define a custom environment for skills with two columns
\newenvironment{skills}{%
    \raggedright%
    \begin{itemize}[label={}, leftmargin=*]%
}{%
    \end{itemize}%
}

% Use the custom twocolumnskills environment for the skills section
\begin{skills}
    \item \skillitem{Python (Pandas)} \skillitem{HTML/CSS} \skillitem{C++} \skillitem{Java}
\end{skills}%

\divider

\begin{skills}
    \item \skillitem{Git/GitHub} \skillitem{MATLAB} \skillitem{LaTeX} \skillitem{Typst}
\end{skills}

\end{paracol}

\end{document}
